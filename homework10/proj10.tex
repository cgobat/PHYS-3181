\documentclass{article}

\usepackage{graphicx, xcolor}
\usepackage{amsmath, amssymb}
\usepackage{float}
\usepackage{enumitem}
\usepackage[colorlinks=true,allcolors=blue]{hyperref}

\usepackage[margin=1in]{geometry}

\def\hwtitle{Homework 10: Laplace equation}
\def\hwauthor{Caden Gobat}
\def\hwdate{\today}

\usepackage{fancyhdr}
\lhead{\hwauthor}
\chead{\hwtitle}
\rhead{\hwdate}
\lfoot{\hwauthor}
\cfoot{}
\rfoot{\thepage}
\renewcommand{\footrulewidth}{0.4pt}
\pagestyle{fancy}

\author{\hwauthor}
\title{\hwtitle}
\date{\hwdate}

\begin{document}

\maketitle
\thispagestyle{fancy}

\section{Introduction}

In this final assignment, we solve the Laplace equation using the Gauss-Seidel method. The Laplace equation is a partial differential equation

\section{Results}

\bigskip
\noindent{\bf Question 1}
\medskip

We can approximate this by first representing the edges in polar coordinates. The central pipe has a radius of 10 cm, and can thus be represented simply by $r(\theta) = 10$. We will use only the upper quarter arc, as it is the same for all four sides of the square. We can represent the top edge of the square (side length 30 cm) using the formula $r(\theta)=\frac{15}{\sin\theta}$, with both of these functions on the domain $\frac{\pi}{4}\leq \theta \leq \frac{3\pi}{4}$. This means the distance between the two on this domain as a function of angle is $d(\theta)=\frac{15}{\sin\theta}-10$. We can integrate this and divide by the width of the domain to find the average: \begin{gather*}
    \int_{\pi\over 4}^{3\pi\over 4} \left(\frac{15}{\sin\theta}-10\right)d\theta \cong 10.73 \\
    \langle d \rangle \cong 10.73\left/\frac{\pi}{2}\right. \cong 6.83\text{ cm}
\end{gather*}

The temperature differential between the center and the edge is 200 K. Therefore, if the gradient were constant, we would expect it to have a value of $\displaystyle \frac{200\text{ K}}{6.83\text{ cm}}=29.27\text{ K/cm}$.

We can the calculate the power loss as \begin{align*}
    P &= -t\sigma \cdot \oint d\mathbf{l}_\perp \cdot \nabla T \\
    &= -(1\text{ cm})(10\text{ W/m/K}) \cdot 2\pi r_\text{contour} \cdot (29.7\text{ K/cm}) \\
    &\approx \boxed{-190\text{ W}}
\end{align*}

\bigskip
\noindent{\bf Question 2}
\medskip

\begin{enumerate}[label=\alph*)]
    \item c
\end{enumerate}

\bigskip
\noindent{\bf Question 3}
\medskip



\bigskip
\noindent{\bf Question 4}
\medskip



\section{Conclusions}

foo

\end{document}