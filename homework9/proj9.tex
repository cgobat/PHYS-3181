\documentclass{article}

\usepackage{graphicx, xcolor}
\usepackage{amsmath, amssymb}
\usepackage{float}
\usepackage[colorlinks=true,allcolors=blue]{hyperref}

\usepackage[margin=1in]{geometry}

\def\hwtitle{Homework 9: Metropolis algorithm and the canonical ensemble}
\def\hwauthor{Caden Gobat}
\def\hwdate{\today}

\usepackage{fancyhdr}
\lhead{\hwauthor}
\chead{\hwtitle}
\rhead{\hwdate}
\lfoot{\hwauthor}
\cfoot{}
\rfoot{\thepage}
\renewcommand{\footrulewidth}{0.4pt}
\pagestyle{fancy}

\author{\hwauthor}
\title{\hwtitle}
\date{\hwdate}

\begin{document}

\maketitle
\thispagestyle{fancy}

\section{Introduction}

Last week, we made a foray into molecular dynamics by implementing an $N$-body simulation of gas particles interacting via only the Lennard-Jones potential (microcanonical ensemble): \begin{equation}
   V(r)=4\epsilon\left[\left(\frac{r_0}{r}\right)^{12} - \left(\frac{r_0}{r}\right)^6\right]
\end{equation}

Now we will use the Metropolis algorithm to simulate the canonical ensemble. Here we accept or reject randomly determined updates based on a probability which is dependent on the partition function of the system. The probability distribution of microstates is given by \begin{equation}
   P(s)=\frac{1}{Z}e^{\frac{-1}{kT}E(s)}
\end{equation}
where $\displaystyle E(s)=\frac{1}{2}\sum_{i=1}^{N} v_{i}^2 + \sum_{i=1}^{N} \sum_{j=i+1}^{N} V(r_{ij})$.

\section{Results}

\bigskip
\noindent{\bf Question 1}
\medskip


\bigskip
\noindent{\bf Question 2}
\medskip


\bigskip
\noindent{\bf Question 3}
\medskip


\bigskip
\noindent{\bf Question 4}
\medskip



\section{Conclusions}

Although a bit tricky on the coding side, I found this assignment to be a nice look into these techniques that are so common in scientific data analysis. The first few problems were a good conceptual introduction to the kind of thinking necessary to complete the more advanced ones.

I do wonder if there are more advanced methods one can employ that are more effective at dealing with large errors in datasets, and would be interested to learn more about this.


\end{document}
