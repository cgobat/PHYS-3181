\documentclass{article}

\usepackage{graphicx, xcolor}
\usepackage{amsmath, amssymb}
\usepackage{float}
\usepackage[colorlinks=true,allcolors=blue]{hyperref}

\usepackage[margin=1in]{geometry}

\def\hwtitle{Homework 7: The Bootstrap Method}
\def\hwauthor{Caden Gobat}
\def\hwdate{\today}

\usepackage{fancyhdr}
\lhead{\hwauthor}
\chead{\hwtitle}
\rhead{\hwdate}
\lfoot{\hwauthor}
\cfoot{}
\rfoot{\thepage}
\renewcommand{\footrulewidth}{0.4pt}
\pagestyle{fancy}

\author{\hwauthor}
\title{\hwtitle}
\date{\hwdate}

\begin{document}

\maketitle
\thispagestyle{fancy}

\section{Introduction}

In this assignment, address the ubiquitous problem of fitting an analytical function to discrete/numerical data. To do this, we will attempt to minimize a $\chi^2$ (chi-squared) goodness-of-fit measure to optimize a set of coefficients on a polynomial. The $\chi^2$ statistic is defined as \begin{equation*}
    \chi^2 \equiv \sum_i \frac{(P(x_i)-y_i)^2}{\sigma_i^2}
\end{equation*}
Where $P(x)$ is a fitted polynomial function and $x$, $y$, and $\sigma$ are the independent quantity, dependent quantity, and standard error in $y$, respectively. We will use this value to determine whether certain models appropriately describe our data.

\section{Results}

\bigskip
\noindent{\bf Question 1}
\medskip

Fitting the first 6 seconds of the data ($0\leq t < 6$) yields a fit of \begin{equation*}
    y(t) = -5.0535t^2 + 2.46 t + 16381.7
\end{equation*}
With a $\chi^2$ value of approximately 6.35. Because we know the equation of motion should be of the form $\displaystyle \frac{1}{2}gt^2 + v_0 t + y_0$, we can interpret these parameters to find that the fit has returned a value for gravitational acceleration of $g\approx-10.11\text{ m/s}^2$.

Given that this subset of the data consists of 12 data points (two samples per second; 6 seconds) and we are fitting 3 parameters (coefficients on the quadratic, linear, and constant terms), the total degrees of freedom is 9. Thus the reduced chi-squared value (i.e. $\frac{\chi^2}{\text{dof}}$) for this fit is 0.706, which (because it is less than 1) indicates that this routine has \emph{over}fit the data.

The percent confidence $\alpha$ for this fit can be calculated by \begin{equation*}
    \int_{\chi_\nu^2}^\infty f(\chi^2) d\chi^2 = \alpha
\end{equation*} where $f(\chi^2)$ is the chi-squared distribution.\footnote{$\displaystyle f(\chi^2) = \frac{e^{-\chi^2/2}}{2^{\nu/2}\ \Gamma(\nu/2)}\cdot\left(\chi^2\right)^{\frac{\nu}{2}-1}$} This $\alpha$ is then the probability that any deviation in $\chi_\nu^2$ from its expected value (based on the known $\chi^2$ distribution) is due to a probabilistic fluke and not an actual flaw in the model/fit parameters. In \texttt{Mathematica}, this calculation can be accomplished by running \verb|1-CDF[ChiSquareDistribution[|$\nu$\verb|], |$\chi^2$\verb|]| (with $\chi^2 \cong 6.35$ and $\nu=N-3=9$), which yields $\alpha\cong0.704$. This means we can be approximately 70\% confident that this model is appropriate for these data.

\begin{table}[H]
    \centering
    \begin{tabular}{c|l|l}
         & Value & Expectation \\
        \hline
        $g$ & -10.11 & -9.81 \\
        $v_0$ & 2.46 & $\sim-15.62$ \\
        $y_0$ & 16381.7 & 16381.567 \\
        $\chi^2$ & 6.35 & 9 \\
        $\chi_\nu^2$ & 0.706 & 1 \\
        conf & 0.704 &
    \end{tabular}
    \caption{Summary of second-order polynomial fit results for the first 6 seconds of \texttt{cannonball.dat} data.}
    \label{tab:fit1}
\end{table}

\bigskip
\noindent{\bf Question 2}
\medskip

Using all 118 data points in the \texttt{cannonball.dat} file yields a best-fit of \begin{equation*}
    y(t) = -4.6924t^2 - 3.4245t + 1.6395.64
\end{equation*}
With $\chi^2 \cong 140.24$. This fit has found $g\approx9.385\text{ m/s}^2$, which is now an \emph{under}estimate compared to its actual known value of $\sim9.81\text{ m/s}^2$. Given that the number of data points for this fit is now $N=118$, we now have $N-3=115$ degrees of freedom, making our reduced chi-squared $\frac{\chi^2}{\text{dof}} = \frac{140.24}{115} \cong 1.22$. This can now be said to be an appropriate fit, as it is greater than but on the order of 1 ($\chi_\nu^2 = 1$ indicates a perfect fit).

The confidence (as described previously) is $\alpha=0.055$, meaning we can only be 5.5\% sure that our model represents a good fit to these data. This could be due to the effects of air resistance, which introduces another term into the underlying differential equation and makes the acceleration non-constant. In the first 6 seconds, the effect would not be noticeable, but over the entire trajectory, it becomes noticeable and thus affects the goodness of the fit.

\begin{table}[H]
    \centering
    \begin{tabular}{c|l|l}
         & Value & Expectation \\
        \hline
        $g$ & -9.385 & -9.81 \\
        $v_0$ & -3.42 & $\sim-15.62$ \\
        $y_0$ & 16395.6 & 16381.567 \\
        $\chi^2$ & 140.24 & 115 \\
        $\chi_\nu^2$ & 1.22 & 1 \\
        conf & 0.055 &
    \end{tabular}
    \caption{Summary of second-order polynomial fit results for the entirety of the \texttt{cannonball.dat} data.}
    \label{tab:fit1}
\end{table}

\bigskip
\noindent{\bf Question 3}
\medskip

An alternative formula for calculating a $\chi^2$ statistic is \begin{equation*}
    \chi^2 = \sum_{i=1}^N \frac{(\bar{x_i}-x_i)^2}{\bar{x_i}}
\end{equation*}
which does not rely on the data coming with any error $\sigma$. However, being able to calculate a chi-squared value for these data does not mean that we will necessarily get a good fit. Radioactive decay behaves exponentially, not as a polynomial, so we have more fundamental issues with our methodology. Second- and third- order polynomials do not even come close visually to fitting the data. A fourth-order polynomial with coefficients \begin{equation*}
    0.0193t^4 - 3.931t^3 + 295.364t^2 - 9923.637t + 131394.55
\end{equation*}
comes within the realm of feasibility, providing the fit shown in Fig.~\ref{fig:decay}. This yields $\chi^2=1606.5$ and we have 20 data points in \texttt{decay.dat}, so there are $20-5=15$ degrees of freedom in this fit. Therefore the reduced $\chi^2$ is $\chi_\nu^2 \cong 107.1$. This is still very poor. There is a chance that moving up to higher order polynomials may improve this, but a much better option would be to change the underlying model itself. For example, using an exponential model with fitted parameters \begin{equation*}
    133565e^{-0.086t}
\end{equation*}
gives $\chi^2=12.2$ with $\chi_\nu^2 = 0.677$. This is an improvement of two orders of magnitude, and requires fewer parameters to attain a much better fit. Therefore we can conclude that this is a better fundamental model. 

We can determine the half life by solving $\displaystyle 0.5 = e^{-0.086t}$. This yields $\ln(0.5)=-0.086t \implies t = 8.06$ days, or around 696000 seconds. This matches most closely with iodine-131, which has a half life of approximately 693 ks.

\begin{figure}[H]
    \centering
    \includegraphics[width=6in]{homework6/decay.png}
    \caption{Radioactive decay data with two models fitted. Parameters for each model are those given above, with the exponential fit being decidedly better.}
    \label{fig:decay}
\end{figure}

\section{Conclusions}

I greatly enjoyed this assignment, and I have a much better understanding of the mechanism behind $\chi^2$ analysis now, having built a fitting engine from the ground up. I also appreciated learning the underlying statistical theory of confidence levels and distributions.

I am curious about how one might deal with errors in the horizontal direction (i.e. an error on the $x$-values as well), and am looking forward to continuing to work on these kinds of problems.

\end{document}
